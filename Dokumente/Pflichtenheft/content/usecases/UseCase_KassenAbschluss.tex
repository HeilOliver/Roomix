%%
%% Author: Oliver Heil
%% 09.03.2018
%%


\documentclass[./detailed_overview_usecases.tex]{subfiles}
\begin{document}

    \subsection{Kassa Abschluss}
    \subsubsection{Detaillierte Benutzungsfallbeschreibungen}
    \textbf{Primary Actor:}
    \begin{itemize}
        \item [-] Back-Office Personal
    \end{itemize}
    \\
    \textbf{Stakeholder and Interests:}
    \begin{itemize}
        \item[-] Back-Office Personal: möchte einen Kassenabschluss
        machen, um den Sollbetrag gegen seine Kasse zu überprüfen und die Kasse für
        die nächste Person vorzubereiten.
        \item[-] Hotelmanager: möchte, dass die korrekten Bestände in
        den Kassen vorhanden sind und möchte über das vorhandene Barvermögen
        Kenntnis haben.
    \end{itemize}

    \subsubsection*{Preconditions}
     -

    \subsubsection*{Postconditions}
    Eine Kasse ist geschlossen und der errechnete Bestand muss mit dem
    Barbestand übereinstimmen. Außerdem werden alle vorhandenen Schecks und
    Kreditkartenabrechnungen der Buchhaltung übergeben.
    Die Kasse enthält nur den in den Stammdaten hinterlegten Betrag.

    \subsubsection*{Main Success Scenario}
    \begin{enumerate}
        \item Der/Die Anwender/In markiert die Kasse im System als geschlossen
        \item Das System errechnet den Sollzustand der Kasse und stellt eine druckbare
        Version davon zu Verfügung. Zusätzlich werden Informationen über Schecks und Kreditkarten
        aufgeführt.
        \item Der/Die Anwender/In vergleicht denn errechneten Soll-Betrag mit dem
        Ist-Betrag. Außerdem werden Schecks- und Kredidkartenrechnungen der Buchhaltung übergeben.
        \item Der/Die Anwender/In füllt oder leert die Kasse auf den in den Stammdaten hinterlegten Betrag.
    \end{enumerate}

    \subsubsection*{Extensions}
    \begin{enumerate}
        \item
        \begin{itemize}
            \item[a.] Soll- und Ist-Betrag stimmen nicht überein
            \begin{itemize}
                \item[i.] Der Ist-Betrag muss im System eingegeben werden und der Fehler
                muss zur Verrechnung der Buchhaltung mitgeteilt werden.
            \end{itemize}
        \end{itemize}
    \end{enumerate}

\end{document}