%%
%% Author: Oliver Heil
%% 09.03.2018
%%


\documentclass[./detailed_overview_usecases.tex]{subfiles}
\begin{document}

    \subsection{Kassen Abschluss}
    \subsubsection{Detaillierte Benutzungsfallbeschreibungen}
    \textit{Primary Actor:}
    Back-Office Personal
    \\
    Stakeholder and Interests:
    \begin{itemize}
        \item[-] Back-Office Personal: möchte einen Kassenabschluss
        machen um den Soll Betrag gegen seine Kasse zu überprüfen.
        \item[-] Hotelmanager: möchte das die korrekten Bestände in
        den Kassen vorhanden sind und möchte über das vorhandene Barvermögen
        kenntnis haben.
    \end{itemize}

    \subsubsection*{Preconditions}
    Eine Person des Back-Offices die die benötigte Berechtigungsstufe
    aufweist um einen Kassenabschluss vorzunehmen.

    \subsubsection*{Postconditions}
    Eine Kasse ist geschlossen und der errechnete bestand muss mit dem
    Barbestand übereinstimmen.

    \subsubsection*{Main Success Scenario}
    \begin{enumerate}
        \item Der/Die Anwender/In markiert die Kasse im System als Geschlossen
        \item Das System errechnet soll zustand der Kasse und stellt eine druckbare
        version davon zu verfügung
        \item Der/Die Anwender/In vergleicht denn errechneten Soll Betrag mit dem
        Ist Betrag
    \end{enumerate}

    \subsubsection*{Extensions}
    \begin{enumerate}
        \item Soll und Ist Betrag stimmen nicht überein
        \begin{itemize}
            \item[a.] Der Ist Betrag muss im System eingegeben werden und der Fehler
            muss zur verrechnung der Buchhaltung mitgeteilt werden.
        \end{itemize}
    \end{enumerate}

    \subsubsection{Sequenz Diagramme}
    \subsubsection{Kontrakte}
\end{document}