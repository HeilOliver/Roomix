%%
%% Author: Robert
%% 08.03.2018
%%

\documentclass[./detailed_overview_usecases.tex]{subfiles}
\begin{document}

    \subsection{Rechnungsposition stornieren}
    \subsubsection{Detaillierte Benutzungsfallbeschreibungen}
    \textit{Primary Actor: Front/Back-Office Personal}
    % describe actor here
    \\
    \textit{Stakeholder and Interests:}
    \begin{itemize}
        \item[-] Front/Back-Office Personal: Möchte in einer Rechnung eine Position stornieren, da ein Fehler aufgetreten ist oder die Position nicht mehr relevant ist.
        \item[-] Individualgast: Möchte einen Fehler korrigiert, oder eine Position gestrichen haben.
    \end{itemize}

    \subsubsection*{Preconditions}

    \subsubsection*{Postconditions}
    Die Rechnungsposition ist storniert und befindet sich nicht mehr in der Rechnung.

    \subsubsection*{Main Success Scenario}
    \begin{enumerate}
        \item Das Front/Back-Office Personal oder der Individualgast findet einen Fehler in der Rechnung, beziehungsweise das Front/Back-Office Personal wird darüber informiert, dass die Rechnungsposition nicht mehr gültig ist.
        \item Das System storniert die bereits gelegte Rechnungposition.
    \end{enumerate}

    \subsubsection*{Extensions}
    \begin{enumerate}
        \setcounter{enumi}{3}
        \item Fehler:
        \begin{itemize}
            \item[a.] Die Gesamtrechnung ist nicht korrekt.
            \begin{itemize}
                \item[i.] Es wird die gesamte Rechnung storniert (UseCase: RechnungStornieren).
            \end{itemize}
        \end{itemize}
    \end{enumerate}
    \subsubsection{Sequenz Diagramme}
    \subsubsection{Kontrakte}
\end{document}