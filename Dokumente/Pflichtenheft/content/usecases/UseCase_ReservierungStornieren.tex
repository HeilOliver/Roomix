%%
%% Author: Oliver Heil
%% 08.03.2018
%%


\documentclass[./detailed_overview_usecases.tex]{subfiles}
\begin{document}

    \subsection{Reservierung stornieren}
    \subsubsection{Detaillierte Benutzungsfallbeschreibungen}
    \textbf{Primary Actor:}
    \begin{itemize}
        \item[-]  Front/Back-Office Personal
    \end{itemize}

    \\
    \textbf{Stakeholder and Interests:}
    \begin{itemize}
        \item[-] Front/Back-Office Personal: Stornierte Reservierungen sollen gelöscht werden.
        \item[-] Gast: Möchte bei Reiseplanänderung von einer Reservierung zurücktreten.
    \end{itemize}

    \subsubsection*{Preconditions}
    Eine gültige Reservierung.

    \subsubsection*{Postconditions}
    Die Reservierung wurde systemintern storniert und belegt keine Hotel Ressourcen mehr.

    \subsubsection*{Main Success Scenario}
    \begin{enumerate}
        \item Der Gast wählt einen Stornierungsgrund aus.
        \item Der Gast bestätigt die Stornierung.
        \item Das System speichert die Stornierung.
        \item Das Front/Back-Office Personal zahlt bereits getätigte Anzahlungen aus.
    \end{enumerate}

    \subsubsection*{Extensions}
    \begin{enumerate}
        \item
            \begin{itemize}
                \item[a.] Der Gast möchte nach einer Stornierung die Reservierung trotzdem wahrnehmen.
                    \begin{itemize}
                        \item[i.] Das Front/Back-Office Personal muss eine neue Reservierung erstellen.
                    \end{itemize}
            \end{itemize}
        \item
            \begin{itemize}
                \item[a.] Die Reservierung ist außerhalb des Stornierungsfrist.
                    \begin{itemize}
                        \item[i.] Der Gast muss eine Stornierungsgebühr bezahlen.
                    \end{itemize}
            \end{itemize}
    \end{enumerate}

\end{document}