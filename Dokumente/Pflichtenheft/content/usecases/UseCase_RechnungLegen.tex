%%
%% Author: Robert
%% 08.03.2018
%%

\documentclass[./detailed_overview_usecases.tex]{subfiles}
\begin{document}

    \subsection{Rechnung legen}
    \subsubsection{Detaillierte Benutzungsfallbeschreibungen}
    \textit{Primary Actor: Front/Back-Office Personal}
    % describe actor here
    \\
    \textit{Stakeholder and Interests:}
    \begin{itemize}
        \item[-] Front/Back-Office Personal: Möchte die Rechnung im System fixieren.
		\item[-] Individualgast: Rechnung auf Korrektheit prüfen.
    \end{itemize}

    \subsubsection*{Preconditions}
    Es muss eine Zwischenrechnung erstellt (UseCase_ZwischenrechnungErstellen) sein.

    \subsubsection*{Postconditions}
    Die Rechnung kann bezahlt werden und die Buchhaltung hat die Rechnung erhalten.

    \subsubsection*{Main Success Scenario}
    \begin{enumerate}
        \item Das Front/Back-Office Personal teilt dem System mit die Rechnung zu legen.
        \item Der Individualgast überprüft die Rechnung.
        \item Das Personal bestätigt die Legung der Rechnung.
    \end{enumerate}

    \subsubsection*{Extensions}
    \begin{enumerate}
        \setcounter{enumi}{3}
        \item Fehler:
        \begin{itemize}
            \item[a.] Dem Individualgast fällt ein Fehler auf.
            \begin{itemize}
                \item[i.] Die gelegte Rechnung muss storniert werden (UseCase_RechnungStornieren).
            \end{itemize}
        \end{itemize}
    \end{enumerate}
    \subsubsection{Sequenz Diagramme}
    \subsubsection{Kontrakte}
\end{document}