%%
%% Author: Oliver Heil
%% 08.03.2018
%%


\documentclass[./detailed_overview_usecases.tex]{subfiles}
\begin{document}

    \subsection{Gast/Kunde Anlegen}
    \subsubsection{Detaillierte Benutzungsfallbeschreibungen}
    \textit{Primary Actor:}
    Front Office Personal
    \\
    Stakeholder and Interests:
    \begin{itemize}
        \item[-] Front Office Personal: Anlegen von neuen Gästen/Kunden in der Gästekartei.
        \item[-] Gast/Kunde: möchte in der Gästekartei sein um schneller Reservierungen oder CheckIn/CheckOut´s vorzunehmen
        und um gegebenenfalls über Neuigkeiten informiert zu werden (Newsletter).
        \item[-] Hotelmanager: möchte alle Gäste gesammelt haben.
    \end{itemize}

    \subsubsection*{Preconditions}
    Eine Person des Front Offices die die benötigte Berechtigungsstufe aufweist um eine änderung in der Gästekartei vorzunehmen.

    \subsubsection*{Postconditions}
    Ein neuer Gast ist in der Gästekartei eingetragen.

    \subsubsection*{Main Success Scenario}
    \begin{enumerate}
        \item Der/Die Anwender/In trägt in das Formular alle Daten des neuen Gastes ein
        \item Das System speichert die eingetragenen Daten in der Gästekartei
    \end{enumerate}

    \subsubsection*{Extensions}
    \begin{enumerate}
        \item Die eingegebenen Daten sind nicht gültig beziehungsweise lückenhaft
        \begin{itemize}
                       \item[a.] Die Daten werden nicht gespeichert
                       \item[b.] Der/Die Anwender/In kann die eingegebenen Daten Korrigieren
        \end{itemize}
        \item Der neu Anzulegende Gast/Kunde ist bereits in der Gästekartei vermerkt
        \begin{itemize}
            \item[a.] Die Daten werden nicht gespeichert
            \item[b.] Der/Die Anwender/In kann die eingegebenen Daten Korrigieren
        \end{itemize}
    \end{enumerate}

    \subsubsection{Sequenz Diagramme}
    \subsubsection{Kontrakte}
\end{document}