%%
%% Author: Robert
%% 08.03.2018
%%

\documentclass[./detailed_overview_usecases.tex]{subfiles}
\begin{document}

    \subsection{Extraleistungen buchen}
    \subsubsection{Detaillierte Benutzungsfallbeschreibungen}
    \textit{Primary Actor: Front-Office Personal}
    % describe actor here
    \\
    \textit{Stakeholder and Interests:}
    \begin{itemize}
        \item[-] Front-Office Personal: schnelle, fehlerfreie und einfache Abwicklung der Buchung von Extraleistungen.
        \item[-] Individualgast: schnelle Buchung von Extraleistungen zu seiner bisherigen Reservierung.
    \end{itemize}

    \subsubsection*{Preconditions}
    Der Individualgast hat bereits eine Reservierung getätigt, oder ist bereits Gast des Hotels
    \subsubsection*{Postconditions}
    %post conditions text
    Die Extraleistung ist auf das jeweilige Zimmer gebucht.

    \subsubsection*{Main Success Scenario}
    \begin{enumerate}
        \item Der Individualgast möchte eine Zusatzleistung auf sein Zimmer buchen.
        \item Das Front-Office Personal gibt die gewünschten Zusatzleistungen in das System ein.
        \item Der Individualgast bestätigt die ausgewählten Leistungen.
        \item Das Front-Office Personal bucht die ausgewählten Zusatzleistungen auf das Zimmer des Kunden.
    \end{enumerate}

    \subsubsection*{Extensions}
    \begin{enumerate}
        \item Keine Reservierung/Check in:
        \begin{itemize}
            \item[a.] Der Individualgast hat vorher keine Reservierung getätigt bzw. ist noch kein Gast des Hotels.
            \begin{itemize}
                \item[i.] Es muss zuerst eine Reservierung erfolgen.
                \item[ii.] Es muss zuerst ein Check in erfolgen.
            \end{itemize}
        \end{itemize}
    \end{enumerate}

    \subsubsection{Sequenz Diagramme}
    \subsubsection{Kontrakte}
\end{document}