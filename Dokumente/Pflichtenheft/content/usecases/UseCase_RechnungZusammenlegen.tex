%%
%% Author: Robert
%% 08.03.2018
%%

\documentclass[./detailed_overview_usecases.tex]{subfiles}
\begin{document}

    \subsection{Rechnung zusammenlegen}
    \subsubsection{Detaillierte Benutzungsfallbeschreibungen}
    \textit{Primary Actor: Front/Back-Office Personal}
    % describe actor here
    \\
    \textit{Stakeholder and Interests:}
    \begin{itemize}
        \item[-] Front/Back-Office Personal: Möchte Rechnungen von mehreren Individualgästen zusammenlegen.
        \item[-] Individualgast: Möchte Rechnungen von anderen Individualgästen übernehmen (Gruppe).
    \end{itemize}

    \subsubsection*{Preconditions}
    Es müssen mehrere Rechnungen erstellt oder eine Rechnung in Teilpositionen geteilt worden sein.

    \subsubsection*{Postconditions}
    %post conditions text
    Die Rechnungen sind zusammengelegt.

    \subsubsection*{Main Success Scenario}
    \begin{enumerate}
        \item Das Individualgast teilt mit, dass er mehrere Rechnungen übernehmen möchte.
        \item Das Personal sucht die gewünschten Rechnungen im System.
        \item Der Individualgast überprüft die Rechnungen.
        \item Das Personal bestätigt die Zusammenlegung.
        \item Das System gibt eine Gesamtrechnung mit den vereinten Positionen aus.
    \end{enumerate}
    \subsubsection{Sequenz Diagramme}
    \subsubsection{Kontrakte}
\end{document}