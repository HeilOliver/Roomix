%%
%% Author: Robert
%% 08.03.2018
%%

\documentclass[./detailed_overview_usecases.tex]{subfiles}
\begin{document}

    \subsection{Rechnung zusammenlegen}
    \subsubsection{Detaillierte Benutzungsfallbeschreibungen}
    \textbf{Primary Actor:}
    \begin{itemize}
        \item [-]  Front/Back-Office Personal
    \end{itemize}
    \\
    \textbf{Stakeholder and Interests:}
    \begin{itemize}
        \item[-] Front/Back-Office Personal: Möchte Rechnungen von mehreren Individualgästen zusammenlegen.
        \item[-] Individualgast: Möchte Rechnungen von anderen Individualgästen übernehmen (Gruppe).
    \end{itemize}

    \subsubsection*{Preconditions}
    Es müssen mehrere Rechnungen erstellt oder eine Rechnung in Teilpositionen geteilt worden sein.

    \subsubsection*{Postconditions}
    Die Rechnungen sind zusammengelegt.

    \subsubsection*{Main Success Scenario}
    \begin{enumerate}
        \item Der Individualgast teilt mit, dass er mehrere Rechnungen zusammenlegen möchte.
        \item Das Front/Back-Office Personal sucht die gewünschten Rechnungen im System.
        \item Der Individualgast überprüft diese Rechnungen.
        \item Das Front/Back-Office Personal bestätigt die Zusammenlegung.
        \item Das System gibt eine Gesamtrechnung mit den zusammengeführten Positionen aus.
    \end{enumerate}

\end{document}