%%
%% Author: Robert
%% 08.03.2018
%%

\documentclass[./detailed_overview_usecases.tex]{subfiles}
\begin{document}

    \subsection{Check-In mit vorhergehender Reservierung}
    \subsubsection{Detaillierte Benutzungsfallbeschreibungen}
    \textit{Primary Actor: Front-Office Mitarbeiter}
    % describe actor here
    \\
    \textit{Stakeholder and Interests:}
    \begin{itemize}
        % zu allgemein ! und heißt es jetzt Front-Office Mitarbeiter oder Rezeptionist?
        \item[-] Rezeptionist: schnelle, fehlerfreie und einfache Abwicklung des CheckIns.
        \item[-] Individualgast: möchte eine schnelle Schlüsselübergabe ohne Komplikationen.
    \end{itemize}

    \subsubsection*{Preconditions}
    Der Individualgast hat bereits eine Reservierung getätigt.
    \subsubsection*{Postconditions}
    %post conditions text
    Der Individualgast erhält die Schlüssel für sein reserviertes Zimmer

    \subsubsection*{Main Success Scenario}
    \begin{enumerate}
        \item Der Individualgast möchte seine Reservierung nun in Anspruch nehmen.
        \item Der Rezeptionist bestätigt die vorliegende Reservierung.
        \item Der Individualgast bezahlt die Anzahlung (UseCase_Akonto buchen)
        \item Der Rezeptionist übergibt den Schlüssel für das passende Zimmer an den Individualgast.
    \end{enumerate}

    \subsubsection*{Extensions}
    \begin{enumerate}
        \item Gast ändert seine Wünsche:
        \begin{itemize}
            \item[a.] Der Individualgast möchte nun ein anderes Zimmer belegen (upgrade)
            \begin{itemize}
                \item[i.] Es wird überprüft ob das gewünschte Zimmer zur Verfügung steht.
                \item[ii.] Der Individualgast wird auf das neue Zimmer gebucht (UseCase WalkIn)
            \end{itemize}
            \item[b.] Der Individualgast möchte seinen Aufenthalt verlängern oder verkürzen
            \item[c.] Der Individualgast möchte Zusatzleistungen buchen
        \end{itemize}
    \end{enumerate}

    \subsubsection{Sequenz Diagramme}
    \subsubsection{Kontrakte}
\end{document}