%%
%% Author: Stefan
%% 18.03.2018
%%

% Preamble
\documentclass[../../Pflichtenheft.tex]{subfiles}
\begin{document}
	Dieses Pflichtenheft beinhaltet folgende Bereiche: \\ \\
	Im Kapitel 2 befinden sich die Stakeholder, welche die Übersicht über
	alle Personen beeinhaltet die mit dem System in Verbindung stehen.
	Die Interessen der Personen als auch welche Verbindung mit dem System
	bestehen wird darunter beschrieben. \\ \\
	Im Kapitel 3 sind die Eigenschaften sowie Funktionen des Systems beschrieben.
	Als erstes werden die verschiedenen Funktionen mit ihrem Nutzen aufgelistet und
	anschließend folgt die detaillierte Ausarbeitung dieser Funktionen.
	Am Ende des Kapitels folgt die Liste mit den Abhängigkeiten und Annahmen, welche
	das System voraussetzt. \\ \\
	Im Kapitel 4 befindet sich das Domänenmodell. Die verschiedenen Klassen und wie
	diese miteinander in Beziehung stehen werden hier beschrieben. 
	Nach dem Übersichtsdiagramm folgen die detaillierten Modelle, welche in
	verschiedene Bereiche aufgeteilt sind.
	Anschließend folgen die einzelnen Beschreibung der Klassen des Modells. \\ \\
	Im Kapitel 5 folgen die relevanten Use Cases. Ein Use Case beschreibt im Detail
	wie eine Funktion des Systems angewendet wird. Anschließend folgen die
	Zustandsdiagramme für wichtige Use Cases. \\ \\
	Im Kapitel 6 befindet sich die Auflistung der nichtfunktionalen Anforderungen.
	Diese Anforderungen sind zwar wichtig, aber im Allgemeinen gültig. \\ \\
	Im Kapitel 7 folgen die Timeboxes. Zuerst sieht man in einer Tabelle das Use Case
	Ranking und anschließend die Timeboxes welche die Reihenfolge der Implementierung
	der Anwendungsfälle darstellen. Diese Anwendungsfälle wurden nach dem Risiko, der
	Architekturrelevanz sowie Benutzerrelevanz bewertet.\\ \\
	Im Kapitel 8 befindet sich das Glossar.
\end{document}